% HC-002-DOC: Entanglement curvature 1D
% Created on 2025-04-14T19:35:45.004753
\documentclass[12pt]{article}
\usepackage{amsmath, amssymb, amsfonts}
\usepackage{graphicx}
\usepackage[hidelinks=true]{hyperref}
\usepackage{cite}
\usepackage{comment}
\usepackage[margin=1in]{geometry}

\begin{document}

\title{A Toy Model Indicating Inverse-Square-Law Emergence from Quantum Entanglement}
\author{The HoloCosmo Project\\HC-002-DOC}
\date{\today}
\maketitle

\begin{abstract}
We present a simplified spin-chain toy model demonstrating that 
a force with an inverse-square radial dependence can, in principle, 
emerge from the decay of mutual information between separated quantum 
subsystems. Although this caricature does not constitute a full theory 
of gravity, it illustrates why entanglement-based approaches to 
emergent gravity are not immediately ruled out. 
\end{abstract}

\section{Introduction}

Proposals that spacetime and gravity could arise from the underlying 
entanglement structure of quantum states have attracted significant 
interest in recent years. In particular, certain holographic frameworks 
have shown promising hints that geometry can be encoded in patterns of 
quantum correlations. However, the leap from these ideas to reproducing 
the familiar $1/r^2$ gravitational potential is far from trivial.

In this note, we outline a minimal “toy model” scenario in which the 
mutual information between two subsystems decays in such a way that 
one can define an ``effective potential'' scaling as $1/r^\alpha$. 
Under the particular condition $\alpha = 1$ (in a one-dimensional 
setup), the associated ``force'' can be made to look like $1/r^2$. 
Although heavily idealized, this suggests that entanglement-driven 
interactions can \emph{mimic} inverse-square forces, at least in 
certain carefully tuned quantum states.

\section{Mutual Information as an Effective Potential}

\subsection{Setup}

Consider a one-dimensional chain of $N$ qubits, labeled by sites 
$x = 1, 2, \dots, N$. Let $A$ and $B$ denote two disjoint blocks 
or ``subsystems'' of qubits, with the distance between the centers 
of $A$ and $B$ denoted by $r$. Define the \emph{mutual information} 
between $A$ and $B$ as
\begin{equation}
    I(A : B) \;=\; S(A) + S(B) - S(A \cup B),
\end{equation}
where $S(\cdot)$ is the von Neumann entropy.

We posit a simple ansatz for the distance dependence of $I(A:B)$, 
namely
\begin{equation}
    I(A:B) \;\sim\; \frac{c}{r^\alpha},
    \label{eq:I_decay}
\end{equation}
for some constant $c > 0$ and exponent $\alpha > 0$. This power-law 
behavior might arise, for example, in a gapless or critical spin system.

\subsection{Defining an Effective Force}

We define an ``entanglement potential'' $V_{\mathrm{ent}}(r)$ as
\begin{equation}
    V_{\mathrm{ent}}(r) \;=\; -\,k \; I(A:B),
\end{equation}
where $k > 0$ is a constant that sets the overall strength. The 
resulting effective force $F(r)$ is then given by
\begin{equation}
    F(r) \;=\; -\,\frac{dV_{\mathrm{ent}}}{dr}
    \;=\; k \,\frac{d}{dr} \Bigl[I(A : B)\Bigr].
\end{equation}
Inserting the ansatz of Eq.~\eqref{eq:I_decay} leads to
\begin{equation}
    I(A:B) \;=\; \frac{c}{r^\alpha}
    \quad \Longrightarrow \quad
    F(r) \;=\; k\,c\,\alpha \;\frac{1}{r^{\alpha + 1}}.
\end{equation}
Thus, if $\alpha + 1 = 2$, i.e.\ $\alpha = 1$, then 
\begin{equation}
    F(r) \;\sim\; \frac{1}{r^2},
\end{equation}
which reproduces the familiar inverse-square dependence of Newtonian 
gravity in this highly stylized scenario.

\section{Interpretation and Caveats}

Although this toy calculation shows how an entanglement-based ``force'' 
can exhibit an inverse-square law, one should note several important 
caveats:

\begin{enumerate}
    \item \textbf{Dimensionality and Tuning.} 
    Achieving $\alpha = 1$ in a genuine spin system typically requires 
    special critical conditions. Moreover, this derivation is in 
    one spatial dimension; real gravitational interactions in 3+1 
    dimensions require more elaborate constructions.

    \item \textbf{Emergent Gravity vs.\ Force Analogy.} 
    Demonstrating an inverse-square force \emph{alone} does not 
    suffice to show that spacetime curvature, geodesic motion, 
    or other hallmarks of gravity truly emerge. 

    \item \textbf{Microscopic Realizations.} 
    Detailed Hamiltonians that yield the exact $I(A:B) \sim 1/r$ 
    fall outside the scope of this short note, but one might look 
    toward certain gapless systems or carefully designed tensor 
    network states to realize (or approximate) such scaling.

    \item \textbf{Deeper Dynamical Constraints.} 
    A fully fleshed-out model must ensure unitarity, consistency 
    with quantum field theory, and a mechanism for incorporating 
    matter fields and local Lorentz invariance. 
\end{enumerate}

\section{Discussion and Outlook}

Despite these restrictions, the simple analysis above indicates that 
the notion of an ``entanglement-induced'' force replicating the 
$1/r^2$ scaling is not obviously inconsistent. Indeed, it illustrates 
why researchers exploring emergent gravity from quantum information 
continue to develop more sophisticated models (e.g.\ using holography 
or tensor networks) in hopes of recovering richer gravitational 
phenomena.

In a more ambitious framework, one would systematically show how 
Einstein-like equations or their generalizations emerge when one 
treats ``entanglement geometry'' in a large-$N$, continuum, or 
holographic limit. The toy example here serves mainly as a sanity 
check that an \emph{inverse-square law from entanglement} is 
conceptually plausible at a basic level.

\section*{Acknowledgments}

We thank all collaborators for discussions on emergent spacetime, 
quantum information, and the interplay between gravity and entanglement.

\end{document}