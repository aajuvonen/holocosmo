% HC-003-DOC: Entanglement curvature 3D
% Created on 2025-04-14T19:42:32.612677
\documentclass[12pt]{article}
\usepackage{amsmath, amssymb, amsfonts}
\usepackage{graphicx}
\usepackage[hidelinks=true]{hyperref}
\usepackage{cite}
\usepackage{comment}
\usepackage[margin=1in]{geometry}

\title{A Toy Model in Three Dimensions Indicating Inverse-Square-Law Emergence from Quantum Entanglement}
\author{The HoloCosmo Project\\HC-003-DOC}
\date{\today}

\begin{document}
\maketitle

\begin{abstract}
We present a simplified extension of a one-dimensional toy model to three dimensions, demonstrating that an effective force with an inverse-square radial dependence may, in principle, emerge from the decay of mutual information between spatially separated quantum subsystems. This work serves as a sanity check to complement previous 1D investigations, highlighting key challenges and potential avenues toward a more realistic emergent-gravity framework.
\end{abstract}

\section{Introduction}
Recent theoretical work has explored the idea that spacetime and gravitational dynamics may emerge from the entanglement properties of an underlying quantum state. In one spatial dimension, a toy model has shown that a power-law decay of mutual information between two subsystems can be associated with an effective force obeying an inverse-square law. This note extends that idea into three dimensions, providing a higher-dimensional generalization and exploring the conditions under which similar inverse-square-law behavior emerges.

\section{Three-Dimensional Setup}
Consider a three-dimensional cubic lattice composed of qubits. Each lattice site is labeled by a coordinate vector
\[
\mathbf{x} = (x_1,x_2,x_3),
\]
with $x_i \in \{1,2,\ldots,L\}$ for a lattice of side length $L$. We assume either open or periodic boundary conditions.

Define two disjoint subsystems, $A$ and $B$, each consisting of a contiguous block (or cluster) of qubits. Let $\mathbf{r}_A$ and $\mathbf{r}_B$ denote the centers of mass of the subsystems, and define the separation
\[
r = \|\mathbf{r}_A - \mathbf{r}_B\|.
\]

The quantum correlations between $A$ and $B$ are quantified by the mutual information,
\[
I(A:B) = S(A) + S(B) - S(A \cup B),
\]
where $S(X)$ is the von~Neumann entropy of subsystem $X$.

\section{Power-Law Decay of Mutual Information}
We postulate that the mutual information decays with the distance $r$ according to a power law:
\[
I(A:B) \sim \frac{c}{r^\alpha},
\]
where $c > 0$ is a constant and $\alpha > 0$ characterizes the decay. In many three-dimensional systems—especially those at criticality or in gapless phases—power-law decays of correlation functions (and thus entanglement measures) are commonplace.

\section{Entanglement Potential and Effective Force}
We define an \emph{entanglement potential} as
\[
V_{\mathrm{ent}}(r) = - k I(A:B),
\]
with $k > 0$ setting the overall coupling strength. Following classical mechanics, we define an effective force as the negative radial derivative of the potential:
\[
F(r) = - \frac{dV_{\mathrm{ent}}}{dr} = k \frac{d}{dr} I(A:B).
\]

Substituting our power-law ansatz,
\[
I(A:B) \sim \frac{c}{r^\alpha},
\]
we compute
\[
F(r) = k \frac{d}{dr} \left( \frac{c}{r^\alpha} \right) 
= -k\,\alpha\,c\,\frac{1}{r^{\alpha+1}}.
\]

\section{Recovering the Inverse-Square Law}
To mimic the Newtonian gravitational force, which scales as $1/r^2$, we require that
\[
\frac{1}{r^{\alpha+1}} \propto \frac{1}{r^2}.
\]
This condition implies:
\[
\alpha + 1 = 2 \quad \Longrightarrow \quad \alpha = 1.
\]
Thus, if the mutual information decays as $I(A:B) \sim 1/r$, the effective force becomes
\[
F(r) \sim \frac{1}{r^2},
\]
which is the familiar inverse-square law observed in Newtonian gravity.

\section{Discussion and Outlook}
This three-dimensional extension represents a straightforward, yet important, step toward more realistic models of emergent gravity:
\begin{itemize}
    \item \textbf{Exponent Tuning:} In a true 3D quantum many-body system, engineering a precise $1/r$ decay may require fine-tuning or may naturally occur only under special critical conditions.
    \item \textbf{Static versus Dynamic Models:} Our analysis is entirely static. For a complete gravitational theory, one must consider time-dependent entanglement correlations and their impact on emergent spacetime dynamics.
    \item \textbf{Towards Continuum Theories:} Ultimately, it is desirable to derive a continuum description linking entanglement dynamics with an effective field theory that includes gravitational degrees of freedom.
\end{itemize}
While this model does not provide a complete theory of gravity, it offers a proof-of-principle that quantum entanglement can encode an effective inverse-square interaction—thereby complementing earlier 1D studies and guiding future research in emergent gravity scenarios.

\section{Conclusion}
We have extended a toy model based on the power-law decay of mutual information from one to three dimensions. By positing that $I(A:B) \sim 1/r$ for well-chosen subsystems in a 3D lattice, the resulting entanglement potential yields an effective force that scales as $1/r^2$. Although highly idealized, this extension reinforces the possibility that gravity-like interactions may emerge from the underlying quantum entanglement structure, and it serves as a stepping stone for more comprehensive explorations connecting quantum information theory with the geometry of spacetime.

\section*{Acknowledgments}
The author thanks the broader community exploring emergent gravity and quantum informational approaches for inspiring this work.

\end{document}